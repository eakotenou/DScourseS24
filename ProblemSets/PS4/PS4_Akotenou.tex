\documentclass[12pt]{article}
\usepackage[margin=1in]{geometry}
\usepackage[all]{xy}


\usepackage{amsmath,amsthm,amssymb,color,latexsym}
\usepackage{geometry}        
\geometry{letterpaper}    
\usepackage{graphicx}

\newtheorem{problem}{}

\newenvironment{solution}[1][\it{Solution}]{\textbf{#1. } }{$\square$}


\begin{document}
\noindent Data Science for Economist section Spring 2024\hfill Problem Set \#4.\\
Emilien Akotenou. (Ph.D Student, OU)

\hrulefill


\begin{section}{Question\#7}
\begin{subsection}{data sources for web scraping}
1. Tell me about some data sources that you would be interested in scraping from.
\newline \textbf{Solution:\newline -Facebook, TikTok, ChatGPT and Google (Demographic Information, follower counts, following counts, and user-generated content, timestamps, engagement metrics (likes, comments, shares, retweets), and hashtags, Geotagged posts and check-ins, Information on sponsored posts, advertisements, and promoted content).
\newline -Sport betting platforms-Xbet.(promotion code, bet, sports statistics, bidder socio-economic characteristics)
\newline -Expenditures information from government annual budget file. 
\newline -Classical texts from Project Gutenberg} \newline \newline
These could be, for example: classical texts from Project Gutenberg, tweets that include a particular hashtag, college or professional sports statistics, financial market
data, etc. For anything you are interested in, there is almost surely data that is freely available on the internet, and most data sources come with highly accessible APIs for R or Python.
In another part of your .tex file, answer the questions raised in the various parts of the previous question.
\end{subsection}
\begin{subsection}{Answer to questions raised in the various parts of the previous question}

\end{subsection}



\end{section}


 
\end{document}
